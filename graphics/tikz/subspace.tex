\documentclass[tikz,border=10pt]{standalone}
\usetikzlibrary{shapes}
\usepackage{python}

% Example how to draw 2D regular sparse grid without boundary points using
% python package and TikZ.
% Note that comments in python part have to be in python syntax.
% compile with --enable-write18

\begin{document}
  \begin{python}
lmax = 6

# calculate points in 1D
pts = []
for l in range(1, lmax + 1):
    lvlPts = []
    for x in range(2 ** l):
        if x % 2 == 1:
            lvlPts.append(x / float(2**l))
    pts.append(lvlPts)

def make_list(size):
    """create a list of size number of zeros"""
    mylist = []
    for i in range(size):
        mylist.append(0)
    return mylist

def make_matrix(rows, cols):
    """
    create a 2D matrix as a list of rows number of lists
    where the lists are cols in size
    resulting matrix contains zeros
    """
    matrix = []
    for i in range(rows):
        matrix.append(make_list(cols))
    return matrix

subspacepts = make_matrix(lmax, lmax)
for x in range (lmax):
    for y in range (lmax):
        subspacepts[x][y] = []
        for a in pts[x]:
            for b in pts[y]:
              subspacepts[x][y].append([a, b])

# epic print debug statement :-)
# print (r"" + repr (subspacepts))

# setup tikz env (scale by lmax to avoid bullets being too close together)
print(r"\begin{tikzpicture}[scale=" + repr(lmax) + "]")


# draw sparse grid nodes
for i in range(lmax):
    for j in range (lmax):
        print (r"\begin{scope}")
        print (r"\pgftransformcm{1}{0}{0}{1}{\pgfpoint{" + repr(i + i*0.25) + "cm}{" + repr(-j - j*0.25) + "cm}}")
        # draw bounding box
        if i + j < lmax:
            print(r"\draw (0, 0) rectangle (1, 1);")
        else:
            print(r"\draw[gray] (0, 0) rectangle (1, 1);")
        for p in subspacepts[i][j]:
            if i + j < lmax:
                print(r"\node at (" + repr(p[0]) + ", " + repr(p[1]) + ") {\\textbullet};")
            else:
                print(r"\node[gray] at (" + repr(p[0]) + ", " + repr(p[1]) + ") {\\textbullet};")
        print (r"\end{scope}")
# close tikz env
print(r"\end{tikzpicture}")
  \end{python}
\end{document}
